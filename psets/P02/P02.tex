\documentclass[11pt,letterpaper]{article}

\usepackage[utf8]{inputenc}
\usepackage[spanish,es-nodecimaldot]{babel}

\usepackage[top=1in, bottom=1in, left=1in, right=1in]{geometry}

\usepackage{xcolor}
\usepackage{color}
\usepackage{graphicx}
\usepackage{minted}

\usemintedstyle{pastie}
\definecolor{MBckg}{HTML}{272822}

\begin{document}

\begin{titlepage}
    \centering

    {\scshape\LARGE Universidad Nacional Autónoma de México \par}

    \vspace{1cm}
    {\scshape\Large Facultad de Ciencias\par}
    \vspace{1.5cm}

    \begin{center}
        \includegraphics[scale=.1]{../../assets/img/logo.png}
    \end{center}

    \vspace{.8 cm}

    {\LARGE Práctica 02: \par}
    {\huge\bfseries Tipos primitivos y bits \par}

    \vspace{0.5cm}
    {\large\itshape Pablo A. Trinidad Paz\par}

    \vfill

    Trabajo presentado como parte del curso de \textbf{Introducción a Ciencias de la Computación}
    impartido por la profesora \textbf{Verónica Esther Arriola Ríos}. \par
    \vspace{0.1cm}
    {\large 27 de agosto de 2018\par}
\end{titlepage}

\textbf{Actividad 2.1} Revisa la documentación de las clases \texttt{Byte},
\texttt{Short}, \texttt{Integer} y \texttt{Long} de \texttt{Java} y revisa
los atributos de clase que permiten acceder a esta información
\textbf{a)} ¿Cuáles encuentras relevantes? \\

Intenta sumarle uno a \texttt{max}, imprime el resultado en base 10 y su
representación en binario. \textbf{b)} Explica qué pasó.

\begin{enumerate}
    \item[a)] Todos los considero relevantes excepto por \texttt{TYPE} y \texttt{BYTES}
    porque creo que son muy redundantes. Respecto a \texttt{BYTES}, ya contamos con
    \texttt{SIZE} el cual menciona el número de bits y respecto a \texttt{TYPE} pues
    también siento muy redundante preguntar de qué clase viene el número si cuando
    lo inicializamos ya lo decimos explicitamente, por ejemplo:
        \begin{minted}[fontsize=\footnotesize,baselinestretch=1,gobble=8]{java}
            double n = Double.MAX_VALUE;
            n.TYPE; // <--- ¿?
        \end{minted}

    \item[b)] El valor de \texttt{max} es $2147483647_{10}$ en base 10 el cuál es equivalente
    a 31 $1s$ precedidos por un $0$ en base 2 ($01111111111111111111111111111111_2$)
    donde el $0$ indica que se trata de un entero positivo y al mismo tiempo se hace uso
    de los 32 bits. Al momento de sumar 1 al valor de \texttt{max}, la representación
    resultante en base 2 se vuelve $10000000000000000000000000000000_2$ la cuál es
    interpretada dentro de Java como un entero negativo debido a que el primer dígito
    indica el signo, y a su vez, el valor de \texttt{max} es equivalente a $-2147483648_{10}$
    en base 10.
\end{enumerate}

\end{document}
