\documentclass[11pt,letterpaper]{article}

\usepackage[utf8]{inputenc}
\usepackage[spanish,es-nodecimaldot]{babel}

\usepackage[top=1in, bottom=1in, left=1in, right=1in]{geometry}

\usepackage{xcolor}
\usepackage{color}
\usepackage{graphicx}
\usepackage{minted}

\usemintedstyle{pastie}
\definecolor{MBckg}{HTML}{272822}

\begin{document}

\begin{titlepage}
    \centering

    {\scshape\LARGE Universidad Nacional Autónoma de México \par}

    \vspace{1cm}
    {\scshape\Large Facultad de Ciencias\par}
    \vspace{1.5cm}

    \begin{center}
        \includegraphics[scale=.1]{../../assets/img/logo.png}
    \end{center}

    \vspace{.8 cm}

    {\LARGE Práctica 03: \par}
    {\huge\bfseries Variables, tipos y operadores \par}

    \vspace{0.5cm}
    {\large\itshape Pablo A. Trinidad Paz\par}
    419004279

    \vfill

    Trabajo presentado como parte del curso de \textbf{Introducción a Ciencias de la Computación}
    impartido por la profesora \textbf{Verónica Esther Arriola Ríos}. \par
    \vspace{0.1cm}
    {\large 29 de agosto de 2018\par}
\end{titlepage}

\begin{enumerate}
    \item ¿Crees que sea posible asignar el valor de un flotante a un entero?
    ¿Cómo crees que funcionaría?

        Si creo que es posible pero debido la misma definición de un número decimal
        dentro de las matemáticas, si la parte decimal es diferente de 0, entonces
        tendría que haber una operación que decida hacer con ella, por ejemplo, redondear
        hacia el siguiente entero, o sólo redondear si el valor excede .5, etc.

    \item Observa el siguiente código:
    \begin{minted}[fontsize=\footnotesize,baselinestretch=1,gobble=8,linenos]{java}
        int a = 1;
        int b = 2;
        int c = 3;
        (a > 3 && ++a <=2) ? b++ : c--;
    \end{minted}
    Sin compilarlo, ¿cuál es el valor final de $a$, $b$ y $c$? Compila y compara
    lo que pensaste con el resultado real. Explica por que cada variable termina
    con el valor que termina.

        \begin{enumerate}
            \item ¿Cuál es el valor final de $a$, $b$ y $c$?
                \begin{itemize}
                    \item $a$ = 1
                    \item $b$ = 2
                    \item $c$ = 2
                \end{itemize}
            \item Explica los valores de cada variable
                \begin{itemize}
                    \item $a$ vale 1 debido a que su valor no es alterado
                    \item $b$ vale 2 debido a que su valor nunca es alterado
                    ya que la condición no se cumplió.
                    \item $c$ vale 3 debido a que la condición se cumplió y
                    su valor sufrió un decremento por 1.
                \end{itemize}
        \end{enumerate}

\end{enumerate}

\end{document}
